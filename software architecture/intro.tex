\section*{مقدمه}
در این مستند به شرح معماری نرم‌افزاری سامانه‌ی آجاره می‌پردازیم. سامانه‌ی آجاره یک سامانه‌ی اجاره‌ی کالا است که در آن کاربران می‌توانند با ایجاد حساب کاربری، کالاهای مختلفی مانند کنسول‌های بازی را اجاره کرده و نظرات خود را در مورد کالاهای مختلف ثبت کنند. \\
در این مستند از روش 
\lr{4 + 1 view}
برای بررسی معماری نرم‌افزار استفاده می‌کنیم. با توجه به این که نرم‌افزار در مرحله‌ی اولیه‌  از پیچیدگی سخت‌افزاری کمی برخوردار است و نیز  پیچیدگی‌های مربوط به استفاده‌ی همروند از سیستم، توسط 
\lr{Framework}‌هایی
مانند
\lr{Django}
قابل حل است، مطابق پیشنهاد
%\ref{TODO} %https://www.cs.ubc.ca/~gregor/teaching/papers/4+1view-architecture.pdf
%\href{https://www.cs.ubc.ca/~gregor/teaching/papers/4+1view-architecture.pdf}{کراچن}
\lr{Krutchen} \LTRfootnote{\href{https://www.cs.ubc.ca/~gregor/teaching/papers/4+1view-architecture.pdf}{https://www.cs.ubc.ca/~gregor/teaching/papers/4+1view-architecture.pdf}}
دید
\lr{Physical}
و
\lr{Process}
در این مستند بررسی نمی‌شوند زیرا در این پروژه مطرح نیستند و تنها دید‌های
\lr{Logical}،
\lr{Development}
و
\lr{Scenario}
مورد بررسی قرار می‌گیرند.\\
در طراحی این معماری از سبک 
\lr{Model View Controller}
یا 
\lr{MVC}
استفاده شده است که با توجه به ماهیت مبنای‌وب پروژه، با استفاده از \lr{Framework}‌هایی مانند 
\lr{Django}، 
باعث تسریع توسعه‌ی نرم‌افزار و سادگی طراحی می‌شود.\\